% INTRODUCTION

LEDs (Light Emitting Diode) are, as the name suggests, a type of electrical component producing light. 
They are widely used due to their low power consumption, long lifetime, small size, and fast switching.
LEDs are made up of strategically layered semiconductors and metals.
Then, in order to produce a working diode, the wafer needs to undergo several process steps.
Some of these can be photolithography, etching, deposition, and annealing.
After, the LEDs should be characterized and tested, e.g. by scanning electron microscopy (SEM), optical microscopy and current-voltage (IV) testing.



% LEDs are important nanotechnology products, which is why making a LED was the lab task in TFE4575.
% While doing this lab the students used many nanotechnology techniques to produce a LED from a metal stack.
% Lithography, etching, deposition, characterization, and more were used to produce a LED.
% These techniques should be familiar for the nanotechnology students specializing in nanoelectronics at NTNU, and is why they were chosen for this lab.
% As stated in the introduction lecture, doing photolithography requires training with failing to be able to do it correctly.
% In almost every single lab session, the students managed to fuck up a big or a small step, resulting in a lot of learning and redoing.
% However, the students managed to produce a working LED in the end, and the lab was a great success.
% The LED will be sold to the highest bidder, or be used as the star in Thords Christmas tree.
% Merry Christmas, we hope you enjoy the read!'
