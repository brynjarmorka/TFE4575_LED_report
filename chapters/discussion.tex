% DISCUSSION

\subsection{Photolithography}
\label{sec:discussion:photolithography}

% lithography undercut 
\noindent All the lithography processes used in this work gave valuable experience.
One of the experiences was how to do a dose test to get a sufficient undercut while using negative photoresist. 
The lab manual for the course \cite{labmanual} states that the undercut should be at least 1 \textmu m deep, while the teaching assistant stated that 0.5-1 \textmu m would be sufficient.
Getting an exact measurement of the undercut was difficult with the Table Top SEM, because the tilting and rotation of the sample in the chamber is very restricted. 
Quantification of the undercut is problematic with the Table Top SEM, but the undercut is clearly visible in the SEM image in \autoref{fig:undercut}.
Estimating the undercut was easier when using the optical microscope. 
First it was assessed whether the slope visible in the optical images in \autoref{fig:undercut_top} and \autoref{fig:undercut_bot} was an undercut or not.
Over- and under focusing on the edge indicated that the slope was an undercut, and this was confirmed in the SEM. 
The quantification of the undercut was done by measuring the length of the slope and comparing this to the width of the finger. 
This is a crude approximation, but it confirmed that the undercut was in the range of 0.5-1.5 \textmu m, assuming that the widest part of the finger was 4 \textmu m wide.


% negative vs positive resist
If all the photolithography steps were done again, using only positive photoresist could have yielded better results.
The first layer made with negative photoresist produced many defect fingers, while the second layer made with positive photoresist on top of the fingers did not have the same defects.
Lift-off with positive photoresist is known to be possible with sufficiently large structures. 
Negative photoresist is normally used for lift-off, but results from this work suggests that trials with positive photoresist should be done for lift-off. 


\subsection{Etch}
\label{sec:discussion:etch}

% the etch curves which are not flat
\noindent Etching of the two different layers gave some artifacts and had as expected different etch rates.
The etch rate for the mesa was as intended higher, because the etch needed to be around 10 times deeper than the etch of the GaAs layer.
Even though the etch solutions were different, it was expected that the mesa etch rate would be higher because the concentration of the etch solution was higher.
Etch solution is not the only factor that affects the etch rate, but it is important.

The plots from the profilometer on the etches show that the surface after the etch is not perfectly flat.
Three artifacts have been identified in the etch process.

One artifact is the contact loss shown in \autoref{fig:Dummy_GaAs_80nm_etch_lost_contact}. 
Here, the depth is measured to 330 nm, but this is uncertain since the contact is lost 20 \textmu m after the etch edge.
The contact loss can be countered with the right settings on the profilometer. 
The artifact could have been something else, but the contact loss is the most likely cause since the profilometer data outside the plot on the left side is flat and correct. 
The plot of the mesa etch could also have this artifact. 
To be sure that the etch depth was deeper than 3.1 \textmu m in the mesa etch, the etch time was increased from the dummy to the LED sample to compensate for eventual profilometer contact loss.

A second artifact are on the vertical edges, where the etch is varying a lot. 
This is illustrated in \autoref{fig:profilometer_GaAs_100nm_etch}.
The other vertical edges varied in other but similar ways. 
This could both be caused by the etch process and the profilometer.
The profilometer might struggle to measure around vertical edges. 
The etch process could be affected by the increased area and by different etch rates on different crystallographic planes.

A third artifact is the squiggly surface, shown between the fingers in \autoref{fig:profilometer_GaAs_100nm_etch}. 
This artifact is caused by the etch process, and is present all over the wafer where the etch was done. 
The surface is varying around $\pm$ 5 nm, which is a lot for a 100 nm etch.
This artifact could potentially block some of the light from the LED, but it is not clear how much of the light this could block.

\subsection{Deposition of passivation layer} 
\noindent If all process steps are done correctly, the LED will emit light at 675 nm.  
For Si$_3$N$_4$, this corresponds to an optical thickness of around 253 nm.
The optical thickness of Si$_3$N$_4$ was measured to be 249.70 nm and calculated to be 247.91 nm.
A few nanometers difference between the measured and calculated thickness is expected, because the $n$ in the calculation is an approximation. 
The values are very close to the target, which is good.

In order to adjust the thickness to get even closer to the target thickness, the deposition time could be adjusted.
The deposition time was 18 minutes and 35 seconds.
This corresponds to a deposition rate of 0.22 nm/sec, assuming that the deposition was linear and that the measured thickness is correct.
Using these numbers, the deposition time could be adjusted to 18 minutes and 50 seconds, which would give an optical thickness of 253.06 nm, slightly closer to target value.

%Adding a passivation layer % https://www.sciencedirect.com/topics/engineering/passivation-layer




\subsection{Surface artifact}
\label{sec:discussion:surface_artifact}

\noindent The surface of the LED sample was far from perfect throughout the process.
While alignment in the MLA was done, multiple surface impurities were visible. 
These impurities are probably what is seen as a high and abrupt peak in the profilometer data in \autoref{fig:profilometer_GaAs_100nm_etch} around 320 \textmu m.
Peaks like this one are visible in all the profilometer data from the sample. 
The optical microscope images in \autoref{fig:led_optical} show that the surface have many surface impurities which have cracked before, during or after annealing. 


The annealing process made the surface artifacts and more visible.
It is most likely that the surface impurities were present before the etch, and that the etch process has made the area around the impurities more susceptible to cracking and slightly different etch rates.
The different etch rates are visible as different colors in the optical microscope images in \autoref{fig:led_optical}, which can arise due to different thicknesses. 
Another argument for the artifacts being a result of the etch process and not the annealing, is that the passivation layer totally covers the surface and protects it from damage. 

\subsection{IV-measurements}

\noindent Initially, only a very low current was detected during the voltage swipe, in the order of pico amperes.
Therefore, the voltage was briefly increased to 10 V.
This caused the current to increase to the maximum allowed current, 30 mA.
This behavior suggests that somewhere in the LED structure, minimum two layers were not in contact.
The increased voltage will anneal the layers and make them diffuse into contact so that current can flow through them.
One possible explanation of this behavior is that during the frontside contact metal deposition, the chamber was exposed to air between the titanium and gold layer.
This can result in an oxide layer forming between them.
An oxide layer can reduce the conductivity, and annealing can fix this problem. 
Annealing is probably what happened to the LEDs when the voltage was set to 10 V, because it fixed the contact problem.

While there are some differences between the IV-curves of the two LEDs, both show a typical diode behavior.
At negative voltages, the LEDs show a low reverse current and one of the LEDs have a breakdown at between -0.5 and -1.0 V.
The biggest difference between the two plotted IV-curves from two different LEDs, is that the 4 um width 40 um period LED does not have a breakdown as the 16 um width 100 um period LED does.
The difference could come from the fact that the 4 um width 40 um period LED was more prone to finger defects, which could cause worse contact between the fingers and the wafer. 
At positive voltages, the two LEDs have the exponential increase in forward current from around 1.3 V.

In \autoref{fig:led_light}, it can clearly be seen that the color of the emitted light from the LED is red.
Red light has a wavelength between 620 and 720 nm \cite{red_light}.
The aim of this work was to make a LED emitting light at 675 nm.
As we observed red light, we can therefore conclude that the target wavelength was roughly achieved.
