% DISCUSSION

\subsection{Photolithography}
\label{sec:discussion:photolithography}

% lithography undercut 
All the lithography processes used in this work gave valuable experience.
One of the experiences was how to do a dose test to get a sufficient undercut while using negative photoresist. 
The lab manual for the course \cite{labmanual} states that the undercut should be at least 1 \textmu m deep, while the teaching assistant stated that 0.5-1 \textmu m would be sufficient.
Getting an exact measurement of the undercut was difficult with the Table Top SEM, because the tilting and rotation of the sample in the chamber is very restricted. 
Quantification of the undercut with the Table Top SEM, but the undercut is clearly visible in the SEM image in \autoref{fig:undercut}.
Estimating the undercut was easier when using the optical microscope. 
First it was assessed whether the slope visible in the optical images in \autoref{fig:undercut_top} and \autoref{fig:undercut_bot} was an undercut or not.
Over- and under focusing on the edge indicated that the slope was an undercut, and this was confirmed in the SEM. 
The quantification of the undercut was done by measuring the length of the slope and comparing this to the width of the finger. 
This is a crude approximation, but it confirmed that the undercut was in the range of 0.5-1.5 \textmu m, assuming that the widest part of the finger was 5 \textmu m wide.


% negative vs positive resist
If all the photolithography steps were done again from start,  positive photoresist would have been preffered ofer negative photoresist to get the best possible results.
All the steps with positive photoresist gave better results than with negative photoresist.
One example of the bad
Negative photoresist is supposed to be better to use for lift-off, because of the undercut, but it was concluded that the positive resist gave sufficiently good results and less edge problems when used in the lift-off process.
One of the students did use positive photoresist for the lift-off process in his project thesis, and found there too that the positive resist gave good enough results.


\subsection{Etch}
\label{sec:discussion:etch}

% the etch curves which are not flat
The plots from the profilometer on the etches show that the etch is not perfectly flat.
Three artifacts have been identified in the etch process, which are the cause of the non-flat etch.

One artifact is the contact loss shown in \autoref{fig:Dummy_GaAs_80nm_etch_lost_contact}. 
Here the depth is measured to 80 nm, but this is uncertain since the contact is lost.
The contact loss can be countered with the right settings on the profilometer. 
The artifact could have been something else, but the contact loss is the most likely cause since the profilometer data outside the plot on the left side is flat and correct. 

A second artifact are on the vertical edges, where the etch is varying a lot. 
This is illustrated in \autoref{fig:profilometer_GaAs_100nm_etch}.
The other vertical edges varied in other but similar ways. 
This could both be caused by the etch process and the profilometer.
The profilometer might struggle to measure around vertical edges. 
The etch process could be affected by  the increased area and by different etch rates on different crystallographic planes.

A third artifact is the squiggly surface, shown between the fingers in \autoref{fig:profilometer_GaAs_100nm_etch}. 
This artifact is caused by the etch process, and is present all over the wafer where the etch was done. 
The surface is varying around $\pm$ 5 nm, which is a lot for a 100 nm etch.
This artifact could potentially block some of the light from the LED, but it is not clear how much of the light this could block.


% \subsection{Deposition of Si$_3$N$_4$}
% \label{sec:discussion:passivation_deposition}



Adding a passivation layer % https://www.sciencedirect.com/topics/engineering/passivation-layer

If all process steps are done correctly, the LED will emit light at 675 nm.  


\subsection{Surface artifact}
\label{sec:discussion:surface_artifact}

The surface of the LED was far from perfect throughout the process.
While alignment in the MLA was done, multiple surface impurities were visible. 
These impurities are probably what is seen as a high and abrupt peak in the profilometer data in \autoref{fig:profilometer_GaAs_100nm_etch} around 320 \textmu m.
Peaks like this one are visible in all the profilometer data from the sample. 
The optical microscope images in \autoref{fig:led_optical} show that the surface have many surface impurities which have cracked before, during or after annealing. 


The annealing process made the surface artifacts and more visible.
It is most likely that the surface impurities were present before the etch, and that the etch process has made the area around the impurities more susceptible to cracking and slightly different etch rates.
The different etch rates are visible as different colors in the optical microscope images in \autoref{fig:led_optical}, which can arise due to different thicknesses. 
Another argument for the artifacts being a result of the etch process and not the annealing, is that the passivation layer totally covers the surface and protects it from damage. 




