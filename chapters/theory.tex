% THEORY

\subsection{LED theory}
\label{LEDtheory}
LEDs are made of a PN junction, which is a semiconductor junction between a p-type and an n-type semiconductor.
The p-type semiconductor is doped with a lot of holes, and the n-type semiconductor is doped with a lot of electrons.
The junction is made by doping the semiconductor with impurities, which are atoms that are not part of the semiconductor crystal.
The impurities are added to the semiconductor to change the electrical properties of the semiconductor.
Light is emitted when an electron jumps from the n-type semiconductor to the p-type semiconductor.
More on semiconductor physics can be found in \cite{streetman2015solid}.


\subsection{Photolithography}
\label{photolithography}

Photolithography is used to make micro- and nanoscale structures on a substrate.
The process is done in a cleanroom, where the air is filtered to remove dust and other particles.
Small contamination can ruin the process by disturbing the resist.
The photoresist is a polymer that is sensitive to light, which will either harden or dissolve depending on the type of photoresist.
This is the steps of the process, and the reason for each step:

% cleaning, spinn coating, soft bake, exposure,  develop, post exposure, hard bake, inpect

\begin{enumerate}
    \item \textbf{Cleaning}: Remove any contamination from the substrate.
    \item \textbf{Spin coating}: Spin coat the photoresist on the substrate.
    \item \textbf{Soft bake}: Bake the photoresist to remove any solvent.
    \item \textbf{Exposure}: Expose certain areas of the photoresist to light. Eventually with mask alignment.
    \item \textbf{Develop}: Develop the photoresist to remove the softened parts.
    \item \textbf{Post exposure bake}: Bake the photoresist to enhance adhesion. \textcolor{red}{??}
    \item \textbf{Hard bake}: Bake the photoresist to remove any solvent. Not often needed.
    \item \textbf{Inspect}: Optical inspect the photoresist to see if it is good.
\end{enumerate}

Negative photoresist are often used for lift-off, because negative photoresist can achieve an undercut which can improve the metallization edges.
Photoresists needs different developing time and exposure doses, and they do change over time when the photoresist is exposed to light, heat, humidity, and time.



\subsection{Etching?}
\label{etching}


\subsection{Characterization equipment}
\label{characterization}

The characterization equipment used in this lab was optical microscope, SEM, profilometer and IES-stuff to measure the LED efficiency.
The theory behind these instuments and how they work are assumed to be known.